\documentclass[notitlepage]{article}
\usepackage[utf8]{inputenc} 
\usepackage{geometry} 		
\usepackage{chngcntr}
\usepackage{amsmath} 			
\usepackage{amssymb}			
\usepackage{mathtools}		
\usepackage{comment} 			
\usepackage{mdframed}			
\usepackage{xcolor}				
\usepackage{fancyhdr}			
\usepackage{listings}			
\usepackage{color}				
\usepackage{tikz}	
\usepackage{tasks}			
\usepackage{exsheets}		
\usepackage{array}			
\usepackage{empheq}
\usepackage{caption}
\usepackage{pdfpages}
\usepackage{tabularx}

\geometry{ 						%Format titlepage (interrupted by newgeometry)
	a4paper,
	total={170mm,257mm}%,
	%left=0mm,
	%top=0mm,
}

%START DEFINE YOUR VARIABLES HERE

\newcommand{\documentName}{Requirement Specification Document}
\newcommand{\projectName}{Label Refinement by Behavioral Similarity}

%END DEFINE YOUR VARIABLES HERE

\title{%
	\documentName\text{ } \\
  \large \projectName\text{ } \\
  }

\author{
	\large \underline{Document owners:}\\ 
	Bianka Bakullari\\
	\texttt{}
	Christopher Beine\\
	\texttt{}
	Nicole Ventsch\\
	\texttt{}
	Juan\\
	\texttt{}
}

\date{\small{Last edited: \today}}

\pagestyle{fancy}
\fancyhf{}
\rhead{}
\lhead{\documentName\space-\space\projectName}

\makeatletter					%Prefix to add ToC to titlepage
\newcommand*{\toccontents}{\@starttoc{toc}}
\renewcommand*\contentsname{}
\makeatother
                  

\begin{document}

\begin{titlepage}
\clearpage\maketitle			%Clear title page
\thispagestyle{fancy}


\end{titlepage}

\title{ \large \textbf{ Contents}}
\tableofcontents

\newpage

\rfoot{\thepage}				%Start printing page-numbers, after title page.

\newgeometry{ 					%Default page formatting on-going #1
	total={170mm,257mm},
	left=20mm,
	top=25mm,
    bottom=30mm					%Causes warning
}

\begin{flushleft}				%Default page formatting on-going #2

\section{Project Drivers}

\subsection{The Purpose of the Project}

The purpose of the project is providing a platform that allows refinements of event logs, where actions that originally had the same label are relabelled if they have different structural behaviours. In order to do so, an interactive web service should be implemented that allows users to upload event logs in the standard XES format and set a threshold for the refinements. This event log should contain actions that are carried out multiple times and are named identically. The web service should then apply an algorithm to it that refines the log in a way that these actions are relabelled based on structural similarity. Finally, the user should be able to download the refined log.

The users can then use the refined log to gain insight by applying process discovery algorithms to the data. By using the refinement algorithm, higher accuracies should be reached and that way lead to a better insight into the processes.

\subsection{The Client, the Customer and other Stakeholders}

The client of this project is the Chair of Process and Data Science at the RWTH Aachen, which provides the supervision and support for this project. Furthermore, there are many different potential users. For example, students or researchers could use the website as a preprocessing step in order to continue research with the refined log. Additionally, business analysts from companies could use it in order to get a better insight from the refined log than from the original one. Since we will provide a web service, it is easily accessible and not limited to be used by a prespecified group of people. 

\section{Project Constraints}

\subsection{Mandated Constraints}

\subsection{Naming Conventions and Definitions}

\subsection{Relevant Facts and Assumptions}


\section{Functional Requirements}

\subsection{The Scope of the Work}

\subsection{The Scope of the Product}

\subsection{Functional and Data Requirements}


\section{Nonfunctional Requirements}

\subsection{Look and Feel Requirements}

\subsection{Usability and Humanity Requirements}

\subsection{Performance Requirements}

\subsection{Operational and Environmental Requirements}

\subsection{Maintainability and Support Requirements}

\subsection{Security Requirements}

\subsection{Cultural and Political Requirements}

\subsection{Legal Requirements}


\section{Project Issues}

\subsection{Open Issues}
Our team has a rough idea of the steps needed to implement the algorithm and the interface.
However, we are unsure of the relative time needed for each of these parts.
Many unexpected bugs in our algorithm could hinder the time we have in our disposal to design a user-friendly web-interface.

While we will try to achieve writing an efficient algorithm, there are no concrete performance measures required.
Drawbacks will probably be found during implementation and decisions will have to be made on the spot.

Since each of the members has to be fully aware of and actively participate in the coding process, it is still unclear how the functionalities will be divided between the group members.
Whether or not some functionalities should be implemented together or individually depends on type and importance of functionality, personal schedules, deadlines and so on.






\subsection{Off-the-Shelf Solutions}
There has already been an implementation of the algorithm in ProM.
This might be helpful in the designing steps of our algorithm.

We will use a code written in Java to automatically generate events logs.

\textcolor{red}{@juan: Where did you get the java code you showed us once?}

\subsection{New Problems}
The Label Refinement algorithm intends to give the user an alternative event log on which process discovery algorithms can be applied.
Whether or not the models resulting from the new refined log have better precision or fitness compared to the original models is beyond our scope.
Since the result also depends on the thresholds and set of candidate activities which are chosen by the user, we assume that the user has some background on Process Mining and has clear intentions for trying to work with a new refined log.
The algorithm does not filter out events or features in the data.

A common mistake the user could make is upload an event log in the .csv or .xlsx format instead of .xes format.






\subsection{Tasks}

\subsection{Migration to the New Product}

\subsection{Risks}
\newpage
\begin{tabularx}{\textwidth}{|X|X|p{3cm}|X|}
\hline
\textbf{Risks} 
&\textbf{Description}
&\textbf{Category}
&\textbf{Mitigation}\\
\hline
Inaccurate expectations. &Stakeholders develop inaccurate expectations (believe that the project will achieve something not in the requirements, plan, etc.).&Stakeholder &Clearly state in the requirement documentation what are the deliverables meant to be done and the scope of the project.\\
\hline
Process inputs are low quality. &Inputs from stakeholders that are low quality (e.g. business case, requirements, change requests). &Stakeholder	&Kindly ask the stakeholder for a more detailed and clearer version of any input they may provide i.e., requirements, business cases.\\
\hline
Misunderstood requirements.	&When requirements are misinterpreted by the project team.	&Communication &Meet with the stakeholders and discuss the requirements again until the team is sure that they have completely understood them.\\ 
\hline
Learning curves. &Project team needs to acquire new skills for the project.	&Team	&Motivate the project team, give them the best practices on the IT field and make experts instruct them using their knowledge and own experience.\\ 
\hline
Integration failure. &Product components will fail to integrate with each other. &Integration	&Establish standards for product development and make sure that the individual components passed flawlessly the unit test.\\ 
\hline
Requirements are incomplete. &Requirements are not fully captured or are overlooked.	&Requirements	&Make a peer-review of the requirement documentation and make sure that nothing is being left out.\\ 
\hline
\end{tabularx}



\subsection{Costs}

\subsection{User Documentation}
\begin{enumerate}
\item Technical documentation: 
\begin{itemize}
  \item Software code documentation.
  \item Technical specifications.
\end{itemize}

\item User documentation including:
\begin{itemize}
\item How to use the UI. 
\item Examples of inputs and outputs.
\item Explanation of error messages.
\item Information to contact the developers (in case of further questions).
\end{itemize}
\end{enumerate}

\subsection{Waiting Room}

\begin{itemize}
\item Additional feature which enables the user to choose a Business Process Model Discovery (BPMD) technique to visualize the resulting process model and to pick the one which is considered to be the best one (according to user's expertise). 
\item Additional feature that allows for the automatic detection of “imprecise labels” by using properties of the Inductive Miner (IM).
\end{itemize}





\addcontentsline{toc}{chapter}{\textbf{References}}
\end{flushleft}
%\bibliography{uw-ethesis}
% Tip 5: You can create multiple .bib files to organize your references. 
% Just list them all in the \bibliogaphy command, separated by commas (no spaces).

% The following statement causes the specified references to be added to the bibliography% even if they were not 
% cited in the text. The asterisk is a wildcard that causes all entries in the bibliographic database to be included (optional).


\begin{thebibliography}{5}
\bibitem{paper}
Lu, Xixi, et al. "Handling duplicated tasks in process discovery by refining event labels." International Conference on Business Process Management. Springer, Cham, 2016.










\end{thebibliography}










\end{document}