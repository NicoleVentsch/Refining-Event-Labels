\documentclass[notitlepage]{article}
\usepackage[utf8]{inputenc} 
\usepackage{geometry} 		
\usepackage{chngcntr}
\usepackage{amsmath} 			
\usepackage{amssymb}			
\usepackage{mathtools}		
\usepackage{comment} 			
\usepackage{mdframed}			
\usepackage{xcolor}				
\usepackage{fancyhdr}			
\usepackage{listings}			
\usepackage{color}				
\usepackage{tikz}	
\usepackage{tasks}			
\usepackage{exsheets}		
\usepackage{array}			
\usepackage{empheq}
\usepackage{caption}
\usepackage{pdfpages}
\usepackage{tabularx}

\geometry{ 						%Format titlepage (interrupted by newgeometry)
	a4paper,
	total={170mm,257mm}%,
	%left=0mm,
	%top=0mm,
}

%START DEFINE YOUR VARIABLES HERE

\newcommand{\documentName}{Requirement Specification Document}
\newcommand{\projectName}{Label Refinement by Behavioral Similarity}

%END DEFINE YOUR VARIABLES HERE

\title{%
	\documentName\text{ } \\
  \large \projectName\text{ } \\
  }

\author{
	\large \underline{Document owners:}\\ 
	Bianka Bakullari\\
	\texttt{}
	Christopher Beine\\
	\texttt{}
	Nicole Ventsch\\
	\texttt{}
	Juan\\
	\texttt{}
}

\date{\small{Last edited: \today}}

\pagestyle{fancy}
\fancyhf{}
\rhead{}
\lhead{\documentName\space-\space\projectName}

\makeatletter					%Prefix to add ToC to titlepage
\newcommand*{\toccontents}{\@starttoc{toc}}
\renewcommand*\contentsname{}
\makeatother
                  

\begin{document}

\begin{titlepage}
\clearpage\maketitle			%Clear title page
\thispagestyle{fancy}


\end{titlepage}

\title{ \large \textbf{ Contents}}
\tableofcontents

\newpage

\rfoot{\thepage}				%Start printing page-numbers, after title page.

\newgeometry{ 					%Default page formatting on-going #1
	total={170mm,257mm},
	left=20mm,
	top=25mm,
    bottom=30mm					%Causes warning
}

\begin{flushleft}				%Default page formatting on-going #2

\section{Project Drivers}

\subsection{The Purpose of the Project}

\subsection{The Client, the Customer and other Stakeholders}

\subsection{Users of the Product}


\section{Project Constraints}

\subsection{Mandated Constraints}

\subsection{Naming Conventions and Definitions}

\subsection{Relevant Facts and Assumptions}


\section{Functional Requirements}

\subsection{The Scope of the Work}

\subsection{The Scope of the Product}

\subsection{Functional and Data Requirements}


\section{Nonfunctional Requirements}

\subsection{Look and Feel Requirements}

\subsection{Usability and Humanity Requirements}

\subsection{Performance Requirements}

\subsection{Operational and Environmental Requirements}

\subsection{Maintainability and Support Requirements}

\subsection{Security Requirements}

\subsection{Cultural and Political Requirements}

\subsection{Legal Requirements}


\section{Project Issues}

\subsection{Open Issues}

\subsection{Off-the-Shelf Solutions}

\subsection{New Problems}

\subsection{Tasks}

\subsection{Migration to the New Product}

\subsection{Risks}

\subsection{Costs}

\subsection{User Documentation and Training}

\subsection{Waiting Room}







\addcontentsline{toc}{chapter}{\textbf{References}}
\end{flushleft}
%\bibliography{uw-ethesis}
% Tip 5: You can create multiple .bib files to organize your references. 
% Just list them all in the \bibliogaphy command, separated by commas (no spaces).

% The following statement causes the specified references to be added to the bibliography% even if they were not 
% cited in the text. The asterisk is a wildcard that causes all entries in the bibliographic database to be included (optional).


\begin{thebibliography}{5}
\bibitem{paper}
Lu, Xixi, et al. "Handling duplicated tasks in process discovery by refining event labels." International Conference on Business Process Management. Springer, Cham, 2016.










\end{thebibliography}










\end{document}